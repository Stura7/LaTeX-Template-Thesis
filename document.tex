% define documentclass
\documentclass[12pt, bibliography=totoc, a4paper, abstracton, numbers=noenddot]{scrreprt}
\usepackage[utf8]{inputenc}
\usepackage[T1]{fontenc}
\inputencoding{utf8}
%compiler anweisung
\nonstopmode
% Absatz - Abstand verändern (Vertikal + Horizontal)
\parskip 12pt
\parindent 0pt
%Seitenumbruch/Absatzkontrolle (Witwe & Waisenkind)
\clubpenalty = 10000 
\widowpenalty = 10000 
\displaywidowpenalty = 10000

%Inhaltsverzeichnis Abstände anpassen
\makeatletter
	\renewcommand{\@pnumwidth}{2em}
	\renewcommand{\@tocrmarg}{3em}
\makeatother

%floatplatzierung
\renewcommand{\floatpagefraction}{.6} 
\renewcommand{\textfraction}{.15}
\renewcommand{\topfraction}{.8} 
\renewcommand{\bottomfraction}{.5} 
\setcounter{topnumber}{3}
\setcounter{bottomnumber}{1} 
\setcounter{totalnumber}{5}
\makeatletter
\setlength{\@fptop}{0pt}
\makeatother
\setcapindent{1em}
% define used packages
% Seitenränder definieren
\usepackage[left=4cm, right=2.0cm, top=3cm, bottom=3cm]{geometry}

\usepackage{graphicx}
\usepackage[ngerman]{babel}
%--------------------------
%schriftart
\usepackage{lmodern}
%--------------------------
\usepackage{listings}


%--------------------------
%custom floats
\usepackage{float}
\floatstyle{plain} % optionally change the style of the new float
\newfloat{Code}{H}{myc}
%--------------------------

%--------------------------
%Glossar
%\input{header/glossar}
%--------------------------

%Literaturverzeichnis
\usepackage[autostyle]{csquotes}
\usepackage[style=alphabetic-verb,
			backend=bibtexu,
			autocite=footnote,
			bibencoding=utf8,
			natbib=true]{biblatex}
\setlength{\bibitemsep}{1em}     % Abstand zwischen den Literaturangaben
\setlength{\bibhang}{2em} 
\defcounter{biburlnumpenalty}{3000}
\defcounter{biburlucpenalty}{6000}
\defcounter{biburllcpenalty}{9000}

%\addbibresource{bibliography.bib}
\bibliography{bibliography.bib}

\usepackage[bottom]{footmisc} %Damit Footnotes immer als Unterstes angezeigt werden, wenn man ein Abbildung mit [b]ottom platziert

%Fußnoten einrücken
\deffootnote[1em]{1.5em}{1em}{\textsuperscript{\thefootnotemark}}

%Fußnotencounter anpassen um durchgängig zu nummerieren
\usepackage{chngcntr}
\counterwithout{footnote}{chapter}

%\usepackage{lastpage}

% advanced tables
\usepackage{array}

% to colorize tablerows

\usepackage[table]{xcolor}

% own colors for tables

\definecolor{lightgray}{gray}{0.9}

\definecolor{middlegray}{gray}{0.7}

% header and footer

%%\usepackage{fancyhdr}
\usepackage{scrpage2}
% set the pagestyle to fancy
\pagestyle{scrheadings}

\clearscrheadfoot
  % define the header
  \ihead{\leftmark}% left header
  \ohead{\HEADER}% right header
  \setheadsepline{0.4pt}% top line
  % define the footer
  \ifoot{\AUTHORNAME{MM}\\\AUTHORNAME{NN}}% left footer
  \ofoot{\pagemark}% right footer
  \setfootsepline{0.4pt} 
  % redefine the chaptermark to have '1. Chaptername' and not 'CHAPTER 1.
  % CHAPTERNAME'
  \renewcommand{\chaptermark}[1]{\markboth{\thechapter.\ #1}{}}
% override the plain style
%\fancypagestyle{plain}{%
%\fancyhf{}% clear all fields
  % define the header
  %\renewcommand{\headrulewidth}{0.0pt}% top line
%
  % define the footer
  %\fancyfoot[L]{\AUTHORNAME{MM}\\\AUTHORNAME{NN}}% left footer
  %\fancyfoot[R]{\pagemark}% right footer
  %\renewcommand{\footrulewidth}{0.6pt}% bottom line
%}


% internal links
\usepackage[colorlinks=true ,linkcolor=black,
			anchorcolor=black ,citecolor=black ,filecolor=black,
			menucolor=black ,urlcolor=black]{hyperref}

% mathematical formulas
\usepackage{amsmath, amssymb}

% to include images side by side
%\usepackage{subfigure}

% define the programming language
\usepackage{listings}
%colors
\definecolor{sh_comment}{rgb}{0.12, 0.38, 0.18 } %adjusted, in Eclipse: {0.25, 0.42, 0.30 } = #3F6A4D
\definecolor{sh_keyword}{rgb}{0.37, 0.08, 0.25}  % #5F1441
\definecolor{sh_string}{rgb}{0.06, 0.10, 0.98} % #101AF9

\lstloadlanguages{Java,sh,bash,Haskell,HTML,PHP,XML}
\lstdefinelanguage{console}{
  morekeywords={},
  otherkeywords={warumgehtdasnicht>,\$}
}
\newcommand{\lstsetconsole}
{ \lstset{language=sh,
        lineskip=-2pt,
        breaklines=true,
        %language=console,
        breaklines=true,
        float=htpb,
        rulecolor=\color{black},        % if not set, the frame-color may be changed on line-breaks within not-black text (e.g. commens (green here))
        commentstyle=\small\color{sh_comment}\textit,
        keywordstyle=\small\color{blue}\bfseries,
        basicstyle=\small\ttfamily,
        stringstyle=\small\color{sh_string}\ttfamily,
        showstringspaces=false,
        frame=single,
        tabsize=2
  }
}
\lstdefinelanguage{scalaconsole}{
  morekeywords={},
  otherkeywords={scala>,\|}
}
\newcommand{\lstsetrepl}
{ \lstset{%language=sh,
        lineskip=-2pt,
        breaklines=true,
        language=scalaconsole,
        breaklines=true,
        rulesepcolor=\color{black},
        commentstyle=\small\color{sh_comment}\textit,
        keywordstyle=\small\color{sh_keyword}\bfseries,
        basicstyle=\small\ttfamily,
        stringstyle=\small\color{sh_string}\ttfamily,
        showstringspaces=false,
        frame=single,
        tabsize=2
  }
}
\newcommand{\lstsetjava}{
 \lstset{language=Java,
	    breaklines=true,
	    rulesepcolor=\color{black},
        commentstyle=\small\color{sh_comment}\textit,
        keywordstyle=\small\color{sh_keyword}\bfseries,
        basicstyle=\small\ttfamily,
        stringstyle=\small\color{sh_string}\ttfamily,
        showstringspaces=false,
        frame=single,
        tabsize=2,
        showspaces=false,
        showtabs=false,
        literate=
        %linewidth=\textwidth,captionpos=b
        %numbers=left, stepnumber=5, numbersep=10pt
 }
}
\lstdefinelanguage{scala}{
  morekeywords={abstract,case,catch,class,def,%
    do,else,extends,false,final,finally,%
    for,forSome,if,implicit,import,lazy,match,mixin,%
    new,null,object,override,package,%
    private,protected,requires,return,sealed,%
    super,this,throw,trait,true,try,%
    type,val,var,while,with,yield},
  otherkeywords={_,:,=,=>,<-,<\%,<:,>:,\#,@},
  sensitive=true,
  morecomment=[l]{//},
  morecomment=[n]{/*}{*/},
  morestring=[b]",
  morestring=[b]',
  morestring=[b]"""
}
\newcommand{\lstsetscala}{
 \lstset{language=scala,
        breaklines=true,
        rulesepcolor=\color{black},
        commentstyle=\small\color{sh_comment}\textit,
        keywordstyle=\small\color{sh_keyword}\bfseries,
        basicstyle=\small\ttfamily,
        stringstyle=\small\color{sh_string}\ttfamily,
        showstringspaces=false,
        frame=single,
        tabsize=2
        %%linewidth=\textwidth,captionpos=b
        %numbers=left, stepnumber=5, numbersep=10pt
 }
}
\newcommand{\lstsethtml}{

 \lstset{language=HTML,
%        alsoletter={<,>,-,!},
        morekeywords={div,section,header,footer,p,a},
        breaklines=true,
        rulecolor=\color{black},        % if not set, the frame-color may be changed on line-breaks within not-black text (e.g. commens (green here))
        commentstyle=\textit\color{green},
        keywordstyle=\bfseries\color{violet},
        basicstyle=\ttfamily\color{black},
        stringstyle=\ttfamily\color{blue},
        emph={id,class,align,type,name,data-role,href},
        emphstyle=\bfseries\color{orange},
        showstringspaces=false,
        frame=single,
        tabsize=2,
        linewidth=\textwidth,captionpos=b,
        numbers=left,stepnumber=1,numbersep=10pt
 }
}
\newcommand{\lstsetphp}{
 \lstset{language=PHP,
        breaklines=true,
        float=htpb,
        rulesepcolor=\color{black},
        commentstyle=\small\color{sh_comment}\textit,
        keywordstyle=\small\color{sh_keyword}\bfseries,
        basicstyle=\small\ttfamily,
        stringstyle=\small\color{sh_string}\ttfamily,
        showstringspaces=false,
        frame=single,
        tabsize=2
        %%linewidth=\textwidth,captionpos=b
        %numbers=left, stepnumber=5, numbersep=10pt
 }
}
%\lstnewenvironment{code}
%    {\minipage{\linewidth}
%    \lstset{}%
%      \csname lst@SetFirstLabel\endcsname}
%    {\csname lst@SaveFirstLabel\endcsname
%    \endminipage}
    
    \lstnewenvironment{code}[1][]%
      {\minipage{\linewidth} 
       \lstset{
       inputencoding=utf8,
             rulecolor=\color{black},        % if not set, the frame-color may be changed on line-breaks within not-black text (e.g. commens (green here))
             commentstyle=\small\color{sh_comment}\textit,
             keywordstyle=\small\color{sh_keyword}\bfseries,
             basicstyle=\small\ttfamily,
             stringstyle=\small\color{sh_string}\ttfamily,
             frame=single,#1}} %wenn listingset leer bleibt erscheint die caption nicht im dokument
      {\endminipage}
    
\newcommand{\lstsethaskell}{
    \lstset{
      language=Haskell,
      rulesepcolor=\color{black},
      commentstyle=\small\color{sh_comment}\textit,
      keywordstyle=\small\color{sh_keyword}\bfseries,
      basicstyle=\small\ttfamily,
      stringstyle=\small\color{sh_string}\ttfamily,
      showstringspaces=false,
      frame=single,
      flexiblecolumns=false,
      basewidth={0.5em,0.45em},
      literate={+}{{$+$}}1 {/}{{$/$}}1 {*}{{$*$}}1 {=}{{$=$}}1
               {==}{{$==$}}2 %{!=}{{$\not\equiv$}}2
               {>}{{$>$}}1 {<}{{$<$}}1 {\\}{{$\lambda$}}1
               {\\\\}{{\char`\\\char`\\}}1
               {->}{{$\rightarrow$} }2 {>=}{{$\geq$}}2 {<-}{{$\leftarrow$}}2
               {<=}{{$\leq$}}2 {=>}{{$\Rightarrow$} }2
               {\ .}{{$\circ$}}2 {\ .\ }{{$\circ$}}2 {(.)}{({$\circ$})}2
               {>>}{{>>}}2 {>>=}{{>>=}}2
               {|}{{$\mid$}}1
    }
}
\lstdefinelanguage{JavaScript}{
  keywords={typeof, new, true, false, catch,%
    function, return, null, catch, switch, var,%
    if, in, while, do, else, case, break},
  ndkeywords={class, export, boolean, throw, implements, import, this},
  sensitive=false,
  comment=[l]{//},
  morecomment=[s]{/*}{*/},
  morestring=[b]',
  morestring=[b]"
}
\lstdefinelanguage{JavaScript}{
  keywords={typeof, new, true, false, catch, function, return, null, catch, switch, var, if, in, while, do, else, case, break},
  keywordstyle=\color{blue}\bfseries,
  ndkeywords={class, export, boolean, throw, implements, import, this},
  ndkeywordstyle=\color{gray}\bfseries,
  identifierstyle=\color{black},
  emph={id},
  emphstyle=\bfseries\color{black},
  sensitive=false,
  comment=[l]{//},
  morecomment=[s]{/**}{*/},
  morecomment=[s]{/*}{*/},
  commentstyle=\small\color{gray}\ttfamily,
  stringstyle=\small\color{red}\ttfamily,
  morestring=[b]',
  morestring=[b]"
}

\newcommand{\lstsetjavascript}{

  \lstset{
     language=JavaScript,
     %backgroundcolor=\color{lightgray},
     extendedchars=true,
     rulecolor=\color{black},        % if not set, the frame-color may be changed on line-breaks within not-black text (e.g. commens (green here))
     basicstyle=\footnotesize\ttfamily,
     showstringspaces=false,
     showspaces=false,
     numbers=left,
     numberstyle=\footnotesize,
     numbersep=9pt,
     tabsize=2,
     breaklines=true,
     showtabs=false,
     captionpos=t
  }
}
\newcommand{\lstsetxml}{
 \lstset{language=XML,
        breaklines=true,
		commentstyle=\small\color{sh_comment}\textit,
        keywordstyle=\small\color{sh_keyword}\bfseries,
        basicstyle=\small\ttfamily,
        stringstyle=\small\color{sh_string}\ttfamily,
        showstringspaces=false,
        frame=single,
        tabsize=2,
        literate=
        %linewidth=\textwidth,captionpos=b
        %numbers=left, stepnumber=5, numbersep=10pt
 }
}

\lstdefinelanguage{CSharp}{
 morekeywords = {abstract,event,new,struct,as,explicit,%
    null,switch,base,extern,object,this,bool,false,%
    operator,throw,break,finally,out,true,byte,fixed,%
    override,try,case,float,params,typeof,catch,for,%
    private,uint,char,foreach,protected,ulong,checked,%
    goto,public,unchecked,class,if,readonly,unsafe,%
    const,implicit,ref,ushort,continue,in,return,using,%
    decimal,int,sbyte,virtual,default,interface,sealed,%
    volatile,delegate,internal,short,void,do,is,sizeof,%
    while,double,lock,stackalloc,else,long,static,%
    enum,namespace,string,partial},
  morecomment = [l]{//},
  morecomment = [l]{///},
  morecomment = [s]{/*}{*/},
  morestring=[b]",
  sensitive = true
}
\newcommand{\lstsetcsharp}{
 \lstset{language=csharp,
        breaklines=true,
        rulesepcolor=\color{black},
        commentstyle=\small\color{sh_comment}\textit,
        keywordstyle=\small\color{sh_keyword}\bfseries,
        basicstyle=\small\ttfamily,
        stringstyle=\small\color{sh_string}\ttfamily,
        showstringspaces=false,
        frame=single,
        tabsize=2
        %%linewidth=\textwidth,captionpos=b
        %numbers=left, stepnumber=5, numbersep=10pt
 }
}
\lstdefinelanguage{FSharp}{
  morekeywords={abstract,and,as,assert,base,begin,%
    class,default,delegate,do,done,downcast,downto,%
    elif,else,end,exception,extern,false,finally,for,fun,%
    function,if,in,inherit,inline,interface,internal,lazy,%
    let,match,member,module,mutable,namespace,%
    new,not,null,of,open,or,override,private,public,rec,%
    return,static,struct,then,to,true,try,type,upcast,use,%
    val,void,when,while,with,yield,asr,land,lor,lsl,lsr,lxor,%
    mod,sig,atomic,break,checked,component,const,%
    constraint,constructor,continue,eager,event,external,%
    fixed,functor,global,include,method,mixin,object,%
    parallel,process,protected,pure,sealed,tailcall,trait,virtual,volatile},     
  sensitive=false,
  morecomment=[l][\color{greencomments}]{///},
  morecomment=[l][\color{greencomments}]{//},
  morecomment=[s][\color{greencomments}]{{(*}{*)}},
  morestring=[b]"
}
\newcommand{\lstsetfsharp}{
 \lstset{language=fsharp,
        breaklines=true,
        rulesepcolor=\color{black},
        commentstyle=\small\color{sh_comment}\textit,
        keywordstyle=\small\color{sh_keyword}\bfseries,
        basicstyle=\small\ttfamily,
        stringstyle=\small\color{sh_string}\ttfamily,
        showstringspaces=false,
        frame=single,
        tabsize=2
        %%linewidth=\textwidth,captionpos=b
        %numbers=left, stepnumber=5, numbersep=10pt
 }
}

%set default pagestyle 
%\pagestyle{empty}

%veringert den abstand zwischen tichpunkten
\usepackage{enumitem}
\setitemize{noitemsep,topsep=0pt,parsep=0pt,partopsep=0pt}

%notwenig für die Definierung von Arrays 
\usepackage{forarray}


% #####
% #
% # START config area
% #
% #####
%\renewcommand{\baselinestretch}{1.50} %zeilenabstand
\newcommand{\HEADER}[0]{Fachhochschule Schmalkalden SS 2011}
\newcommand{\PAGENUMBERS}[0]{Seite \pagemark \ von \pageref{LastContentPage}}
\newcommand{\DATE}[0]{\today}

\DefineArrayVars{,}{/}{:}{;}{@}
 {MM,NN}
 {NAME:
 Max Müller; Nina Neumann/
 MATNR:
 123456; 234567/
 STREET:
 {Musterstraße 11}; {Musterstraße 12}/
 ZIP:
 12345; 12345/
 TOWN:
 Musterstadt; Musterstadt}
 
\newcommand{\AUTHORNAME}[1]{
\csname NAME@#1\endcsname}

\newcommand{\MATNR}[1]{
\csname MATNR@#1\endcsname}

\newcommand{\STREET}[1]{
\csname STREET@#1\endcsname}

\newcommand{\ZIP}[1]{
\csname ZIP@#1\endcsname}

\newcommand{\TOWN}[1]{
\csname TOWN@#1\endcsname}

\newcommand{\REFERENT}[0]{Prof. Dr. Marianne Musterfrau}
\newcommand{\KOREFERENT}[0]{Prof. Dr. Max Mustermann}

\newcommand{\TITLE}[0]{Thema}
\newcommand{\COURSE}[0]{Studiengang}
\newcommand{\TYPE}[0]{Masterarbeit}
\newcommand{\COMPLETION}[0]{Master of Science}
\newcommand{\UNIVERSITYTOWN}[0]{Schmalkalden}

% #####
% #
% # END config area 
% #
% #####

\usepackage{scrhack}
% starting the document
\begin{document}
% set pagenumbering to roman(I II III IV)
\pagenumbering{Roman}
% input the title
% #####
% # This is the titlelayout from Prof. Dr. Oliver Braun (Fachhochschule
% # Schmalkalden)
% #
% # if you want to use it, please comment the lower title layout 
% #  
% #####

% \author{
% \\[1em]
% an der Fachhochschule Schmalkalden\\
% Fakultät Informatik\\[2em]
%   Referent: \REFERENT\\
%   Koreferent: \COREFERENT\\[2em]
% eingereicht von:\\[1em]
% \AUTHOR\\
% Matrikel-Nr.: \MATNR\\
% \STREET\\
% \TOWN\\
% }
% \date{Schmalkalden, den \DATE}
% \title{\TITLE}
% \subtitle{
% \TYPE\\
% Zur Erlangung des akademischen Grades eines\\
% \COMPLETION
% }
%\maketitle

% #####
% #
% # Default layout
% #
% #####
\begin{titlepage}
  \begin{center}
  	\includegraphics[width=0.7\textwidth]{images/Logo_dt}
  \end{center}
  \vspace{40pt}
  \sffamily
  \begin{tabular}{|l>{\raggedright\hspace{0pt}\arraybackslash}p{15cm}}
    & \\
    & \Large\textbf{\TYPE}\\[\baselineskip]
    & \huge\textbf{\TITLE}\\\\
    & \\
  \end{tabular}
  \vfill
  \begin{tabular}{lp{0.4\linewidth}l@{}}
    & \multicolumn{2}{l}{Fakultät Informatik}\\[\baselineskip]
    & \multicolumn{2}{l}{Referent: \REFERENT}\\[\baselineskip]
    & \\
    & erstellt von:			& \\[\baselineskip]
    & \AUTHORNAME{MM}		& \AUTHORNAME{NN}\\[\baselineskip]
    & Matr.-Nr.\MATNR{MM}	& Matr.-Nr.\MATNR{NN}\\[\baselineskip]
    & \STREET{MM} 			& \STREET{NN}\\[\baselineskip]
    & \ZIP{MM}\TOWN{MM} 	& \ZIP{NN}\TOWN{NN}\\[\baselineskip]
    & \\[\baselineskip]
    & \UNIVERSITYTOWN, den \DATE\\[\baselineskip]
  \end{tabular}
\end{titlepage}

\begin{abstract}
Diese Arbeit \ldots 
\end{abstract}


% load the preamble
\input{content/preamble}

% loads the fancy pagestyle for register part
%%% set the pagestyle to fancy
\pagestyle{fancy}

\fancyhf{}% clear all fields
  % define the header
  \fancyhead[L]{\leftmark}% left header
  \fancyhead[R]{\HEADER}% right header
  \renewcommand{\headrulewidth}{0.4pt}% top line

  % define the footer
  \fancyfoot[L]{\AUTHORNAME{MM}}% left footer
  \fancyfoot[R]{\pagemark}% right footer
  \renewcommand{\footrulewidth}{0.6pt}% bottom line

  % redefine the chaptermark to have '1. Chaptername' and not 'CHAPTER 1.
  % CHAPTERNAME'
  \renewcommand{\chaptermark}[1]{\markboth{\thechapter.\ #1}{}}

% override the plain style
\fancypagestyle{plain}{%
\fancyhf{}% clear all fields
  % define the header
  \renewcommand{\headrulewidth}{0.0pt}% top line

  % define the footer
  \fancyfoot[L]{\AUTHORNAME{MM}}% left footer
  \fancyfoot[R]{\pagemark}% right footer
  \renewcommand{\footrulewidth}{0.6pt}% bottom line
}


% create the registers
\tableofcontents
\listoffigures
\addcontentsline{toc}{chapter}{Abbildungsverzeichnis}
\listoftables
\addcontentsline{toc}{chapter}{Tabellenverzeichnis}
\lstlistoflistings
\addcontentsline{toc}{chapter}{Listingverzeichnis}
\newpage
%\glsaddall
%\printglossary
%\printglossary[type=\acronymtype] % prints just the list of acronyms
%\addcontentsline{toc}{chapter}{Glossar}
%\newpage
\newcounter{romanCount}
\setcounter{romanCount}{\value{page}}
% set pagenumbering to arabic(1 2 3 4)
\pagenumbering{arabic}
% loads the fancy pagestyle for main part
%%% set the pagestyle to fancy
\pagestyle{fancy}

\fancyhf{}% clear all fields
  % define the header
  \fancyhead[L]{\leftmark}% left header
  \fancyhead[R]{\HEADER}% right header
  \renewcommand{\headrulewidth}{0.4pt}% top line

  % define the footer
  \fancyfoot[L]{\AUTHORNAME{MM}}% left footer
  \fancyfoot[R]{\PAGENUMBERS}% right footer
  \renewcommand{\footrulewidth}{0.6pt}% bottom line

  % redefine the chaptermark to have '1. Chaptername' and not 'CHAPTER 1.
  % CHAPTERNAME'
  \renewcommand{\chaptermark}[1]{\markboth{\thechapter.\ #1}{}}

% override the plain style
\fancypagestyle{plain}{%
\fancyhf{}% clear all fields
  % define the header
  \renewcommand{\headrulewidth}{0pt}% top line

  % define the footer
  \fancyfoot[L]{\AUTHORNAME{MM}\\\AUTHORNAME{NN}}% left footer
  \fancyfoot[R]{\PAGENUMBERS}% right footer
  \renewcommand{\footrulewidth}{0.6pt}% bottom line
}

% set the pagestyle to fancy
\pagestyle{scrheadings}

\clearscrheadfoot
  % define the header
  \ihead{\leftmark}% left header
  \ohead{\HEADER}% right header
  \setheadsepline{0.4pt}% top line
  % define the footer
  \ifoot{\AUTHORNAME{MM}\\\AUTHORNAME{NN}}% left footer
  \ofoot{\pagemark}% right footer
  \setfootsepline{0.4pt} 
  % redefine the chaptermark to have '1. Chaptername' and not 'CHAPTER 1.
  % CHAPTERNAME'
  \renewcommand{\chaptermark}[1]{\markboth{\thechapter.\ #1}{}}
% override the plain style
%\fancypagestyle{plain}{%
%\fancyhf{}% clear all fields
  % define the header
  %\renewcommand{\headrulewidth}{0.0pt}% top line
%
  % define the footer
  %\fancyfoot[L]{\AUTHORNAME{MM}\\\AUTHORNAME{NN}}% left footer
  %\fancyfoot[R]{\pagemark}% right footer
  %\renewcommand{\footrulewidth}{0.6pt}% bottom line
%}

% #####
% # load the chapter from the files
% #
% # TODO: create new chapter
% #####
\chapter{Chapter eins}

$(\sum_{i=1}^{l}(n_i - 1)) + l - 1 = (\sum_{i=i}^{l} n - 1) - 1$


bla test.\footnote{http://www.fh-schmalkalden.de}

%Das könnte \gls{zb} in einer arbeit stehen.

\lstsethaskell
\begin{code}[label=listinghaskell,caption=This is Haskell]
module Main where

-- this is a comment
f :: Show a => a -> Int -> String
f x i = show x ++ show i

main :: IO ()
main = do
  putStrLn "Hello World"
  putStrLn $ f 1.2 3
  print $ sum [1..10]
\end{code}

Als nächstes kommt Java.%\gls{glo:Java}.


\lstsetjava
\begin{code}[label=listingjava,caption={This is Java}]
// Comment
class Main {
 public static void main(String[] args) {
    System.out.println("Hello World");
  }
}
\end{code}

\lstsetscala
\begin{code}[label=listingscala,caption=This is Scala]
// Comment
class Main extends App {
  println("Hello World")
}
\end{code}

place Code like normal floats
\lstsetjavascript
\begin{Code}
	\centering
	\lstinputlisting[caption={This is javascript},label=listingjavascript,linerange=1-2]{./listings/hello.js}
\end{Code}
\lstsetphp
\begin{code}[label=listingphp,caption={This is PHP}]
<HTML>
<HEAD>
<TITLE> Hello World in PHP </TITLE>
</HEAD>
<BODY>
<?
// Comment
 print("Hello World");
?>
</BODY>
</HTML>
\end{code}

\section{Formeln}

Komplette Referenz zu AMSMath siehe \\
\url{ftp://ftp.ams.org/ams/doc/amsmath/short-math-guide.pdf}

\begin{align}
 \int_{a}^{b} x\,dx
 & = \left.\frac{1}{2} x^2\right\vert_{a}^{b}\\
 & = \frac{1}{2} b^2 - \frac{1}{2} a^2 \\
 \intertext{mit $a=1$ und $b=3$ folgt:}
 \notag
 & = \frac{1}{2} \left(3^2 - 1^2\right)\\
 & = 5
\end{align}

\section{hans wurst}
one more page

\section{hans wurst2}
one more page

\section{hans wurst3}
one more page \autocite[Vgl.][]{braun:scala,fab:vorlage}


\input{content/chapter2}
\label{LastContentPage}
\newpage

% loads the fancy pagestyle for register part
%%% set the pagestyle to fancy
\pagestyle{fancy}

\fancyhf{}% clear all fields
  % define the header
  \fancyhead[L]{\leftmark}% left header
  \fancyhead[R]{\HEADER}% right header
  \renewcommand{\headrulewidth}{0.4pt}% top line

  % define the footer
  \fancyfoot[L]{\AUTHORNAME{MM}}% left footer
  \fancyfoot[R]{\pagemark}% right footer
  \renewcommand{\footrulewidth}{0.6pt}% bottom line

  % redefine the chaptermark to have '1. Chaptername' and not 'CHAPTER 1.
  % CHAPTERNAME'
  \renewcommand{\chaptermark}[1]{\markboth{\thechapter.\ #1}{}}

% override the plain style
\fancypagestyle{plain}{%
\fancyhf{}% clear all fields
  % define the header
  \renewcommand{\headrulewidth}{0.0pt}% top line

  % define the footer
  \fancyfoot[L]{\AUTHORNAME{MM}}% left footer
  \fancyfoot[R]{\pagemark}% right footer
  \renewcommand{\footrulewidth}{0.6pt}% bottom line
}


% #####
% # load the bibliography
% #####


\printbibliography
%%% set the pagestyle to fancy
\pagestyle{fancy}

\fancyhf{}% clear all fields
  % define the header
  \fancyhead[L]{\leftmark}% left header
  \fancyhead[R]{\HEADER}% right header
  \renewcommand{\headrulewidth}{0.4pt}% top line

  % define the footer
  \fancyfoot[L]{\AUTHORNAME{MM}}% left footer
  \fancyfoot[R]{\pagemark}% right footer
  \renewcommand{\footrulewidth}{0.6pt}% bottom line

  % redefine the chaptermark to have '1. Chaptername' and not 'CHAPTER 1.
  % CHAPTERNAME'
  \renewcommand{\chaptermark}[1]{\markboth{\thechapter.\ #1}{}}

% override the plain style
\fancypagestyle{plain}{%
\fancyhf{}% clear all fields
  % define the header
  \renewcommand{\headrulewidth}{0.0pt}% top line

  % define the footer
  \fancyfoot[L]{\AUTHORNAME{MM}}% left footer
  \fancyfoot[R]{\pagemark}% right footer
  \renewcommand{\footrulewidth}{0.6pt}% bottom line
}

\pagenumbering{Roman}
\setcounter{page}{\value{romanCount}}
% #####
% # load the appendix from the files
% #####
\input{content/appendix}

% #####
% # load the sworn declaration
% #####
\chapter*{Eidesstattliche Erklärung}\markboth{Eidesstattliche Erklärung}{}
\addcontentsline{toc}{chapter}{Eidesstattliche Erklärung}
Ich versichere an Eides Statt durch meine eigenhändige Unterschrift, dass ich die vorliegende Arbeit selbstständig und ohne fremde Hilfe angefertigt habe. Alle Stellen, die wörtlich oder dem Sinn nach auf Publikationen oder Vorträgen anderer Autoren beruhen, sind als solche kenntlich gemacht. Ich versichere außerdem, dass ich keine andere als die angegebene Literatur verwendet habe. Diese Versicherung bezieht sich auch auf alle in der Arbeit enthaltenen Zeichnungen, Skizzen, bildlichen Darstellungen und dergleichen.

Die Arbeit wurde bisher keiner anderen Prüfungsbehörde vorgelegt und auch noch nicht veröffentlicht.
\vspace{20mm}

\centering
\begin{tabular}{p{0mm}>{\centering\arraybackslash}p{0.35\linewidth}p{5mm}
>{\centering\arraybackslash}p{0.45\linewidth}p{0mm}}
&\textit{\large \UNIVERSITYTOWN,} &&&\\
& \textit{\large den \today}&&\hrulefill& \\
&\small Ort, Datum&&\small \AUTHORNAME{MM}&
\end{tabular}
\vspace{15mm}

\centering
\begin{tabular}{p{0mm}>{\centering\arraybackslash}p{0.35\linewidth}p{5mm}
>{\centering\arraybackslash}p{0.45\linewidth}p{0mm}}
&\textit{\large \UNIVERSITYTOWN,} &&&\\
& \textit{\large den \today}&&\hrulefill& \\
&\small Ort, Datum&&\small \AUTHORNAME{NN}&
\end{tabular}
\pagebreak
% end of the document
\end{document}
